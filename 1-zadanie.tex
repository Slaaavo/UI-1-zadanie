\documentclass[10pt,twoside,slovak,a4paper]{article}

\usepackage[slovak]{babel}
\usepackage[T1]{fontenc}
\usepackage[utf8]{inputenc}
\usepackage{graphicx}
\usepackage{booktabs}
\usepackage{url} % príkaz \url na formátovanie URL
\usepackage{hyperref} % odkazy v texte budú aktívne (pri niektorých triedach dokumentov spôsobuje posun textu)

\usepackage{cite}
\usepackage{natbib}
\usepackage{times}
\usepackage[dvips,dvipdfm,a4paper,centering,textwidth=14cm,top=4.6cm,headsep=.6cm,footnotesep=1cm,footskip=0.6cm,bottom=3.8cm]{geometry}

\pagestyle{headings}

\title{Nanoboty v medicíne\thanks{Blablabla}} % meno a priezvisko učiteľa na cvičeniach

\author{Jaroslav Sumbal, Andrej Švec\\[2pt]
	{\small Slovenská technická univerzita v Bratislave}\\
	{\small Fakulta informatiky a informačných technológií}\\
	{\small \texttt{sumbal13@fiit.stuba.sk, svec13@fiit.stuba.sk}}\\
	{\small Hlavný zdroj: \cite{Zdroj}}
	}

\date{\small 21. február 2015} % upravte



\begin{document}

\maketitle


\begin{abstract}


\end{abstract}

\section{Problémové prostredie}

Problémove prostredie je ľuďské telo. Je totiž veľmi zratiteľné. Stačí malý pád a zlomenina je na svete. Našťastie, väčšinu týchto fyzických zranení vieme napraviť, alebo si telo pomôže samo. Toto neplatí pre prípad zákernej choroby, akou je rakovina. Čo je to rakovina? Rakovina vznikne vtedy, keď sa niektoré bunky začnú chorobne množiť a brániť zdravým bunkám vykonávať svoju prácu. V dnešnej dobe existuje viacero druhov liečenia rakoviny, ale nefungujú dokonale. Nedajú sa vyliečiť všetky druhy rakoviny a niekedy sa dá len spomaliť ich priebeh. Preto je potrebné nové riešenie, ktoré ponúka pole nanotechnológii.
\\
Ľuďské telo je veľmi komplikované a preto treba veľmi inteligetné riešenie. Je potrebné rozoznať zdravé bunky od tých chorých, aby sa pacientovi náhodou neprihoršilo. Ďalšia nevyhnutnosť je dobrý transportný systém. Telo už taký má, je to krvný obeh. Avšak krvný obeh ponúka plejádu možností, veľa rôznych ciest. Ktorou sa vydať? Táto otázka potrebuje premyslenú odpoveď, aby bola liečba čo najefektívnejšia.
\\
Veľký problém taktiež predstavuje imunitný systém ľuďského tela. Ako presvedčiť telo, že nanoboty sú preň dobré aby ich nezničil? Nie je to vôbec ľahké, keď si uvedomíme, že niekedy je aj transplantovaný orgán odvrhnutý, teda niečo prirodzené telu. A tu je treba aby prijalo niečo syntetické. Je to jeden zo základných problémov, ktorý treba vyriešiť, aby vôbec celý tento výskum mal zmysel.

\section{Opis znalostného konateľa}

\subsection{Ciele}
Nanoboty sú vpustené do organizmu za účelom ozdravenia organizmu a to odstránením rakovinotvorných buniek, prítomných v organizme. Takisto je cieľom, aby bolo minimalizované množstvo zdravých buniek, ktoré boli zničené pri odstraňovaní rakovinotvorných buniek. Teda je dôležité aby pri odstraňovaní chorých buniek neboli poškodené a znižené aj zdravé bunky. Nakoniec ide aj o minimalizáciu času potrebného na nájdenie rakovinotvornej bunky od doby jej vzniku.

\subsection{Vnemy}
Vstupmi pre nanobota sú rôzne chemické látky, ktoré nájde v bunkách a ich koncentrácia, tvar bunky, sekvencia DNA, ktorú vie prečítať.

\subsection{Akcie}
Nanobot musí na základe vnemov vedieť reagovať, a to buď zničením, alebo ponechaním bunky. Rovnako musí vedieť komunikovať ostatným nanobotom, DNA sekvenciu, ktorú našiel v nejakej konkrétnej bunke, aby na základe tejto analýzy zistili, ktorá DNA je nepoškodená.


\section{Znalosti konateľa}
\label{sec:znalosti}
Rakovinotvorné bunky majú oveľa väčšiu spotrebu cukru. Takisto je možné tieto bunky detekovať na základe toho, že majú poškodenú DNA. Nanobot môže vyhľadať a prečítať DNA v bunkovom jadre a na základe toho vyhodnotiť, či je bunka chorá, alebo nie.\cite{Wikipedia-nador,cancer-cell-metabolism}

\section{Správania konateľa}
Nanobot musí byť schopný analyzovať prostredie na základe správania sa rakovinotvorných buniek. Ako bolo opísané v časti \ref{sec:znalosti} rakovinové bunky majú vyššiu spotrebu cukru a takisto majú poškodenú DNA.

\section{Záver}

\listoffigures
\bibliography{zdroje}
\bibliographystyle{unsrt}
\end{document}

