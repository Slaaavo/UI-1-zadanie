\documentclass[10pt,twoside,slovak,a4paper]{article}

\usepackage[slovak]{babel}
\usepackage[T1]{fontenc}
\usepackage[utf8]{inputenc}
\usepackage{graphicx}
\usepackage{booktabs}
\usepackage{url} % príkaz \url na formátovanie URL
\usepackage{hyperref} % odkazy v texte budú aktívne (pri niektorých triedach dokumentov spôsobuje posun textu)

\usepackage{cite}
\usepackage{natbib}
\usepackage{times}
\usepackage[dvips,dvipdfm,a4paper,centering,textwidth=14cm,top=4.6cm,headsep=.6cm,footnotesep=1cm,footskip=0.6cm,bottom=3.8cm]{geometry}

\pagestyle{headings}

\title{Nanoboty v medicíne\thanks{Blablabla}} % meno a priezvisko učiteľa na cvičeniach

\author{Jaroslav Sumbal, Andrej Švec\\[2pt]
	{\small Slovenská technická univerzita v Bratislave}\\
	{\small Fakulta informatiky a informačných technológií}\\
	{\small \texttt{sumbal13@fiit.stuba.sk, svec13@fiit.stuba.sk}}\\
	{\small Hlavný zdroj: \cite{Hlavny}}
	}

\date{\small 21. február 2015} % upravte



\begin{document}

\maketitle


\begin{abstract}


\end{abstract}

\section{Problémové prostredie}

Problémove prostredie je ľuďské telo. Je totiž veľmi zratiteľné. Stačí malý pád a zlomenina je na svete. Našťastie, väčšinu týchto fyzických zranení vieme napraviť, alebo si telo pomôže samo. Toto neplatí pre prípad zákernej choroby, akou je rakovina. Čo je to rakovina? Rakovina vznikne vtedy, keď sa niektoré bunky začnú chorobne množiť a brániť zdravým bunkám vykonávať svoju prácu. V dnešnej dobe existuje viacero druhov liečenia rakoviny, ale nefungujú dokonale. Nedajú sa vyliečiť všetky druhy rakoviny a niekedy sa dá len spomaliť ich priebeh. Preto je potrebné nové riešenie, ktoré ponúka pole nanotechnológii. 

\section{Záver}

\listoffigures
\bibliography{literatura}
\bibliographystyle{unsrt}
\end{document}

